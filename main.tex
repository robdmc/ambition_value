%%%%%%%%%%%%%%%%%%%%%%%%%%%%%%%%%%%%%%%%%%%%%%%%%%%%%%%%%%%%%%%%%%%%%% % LaTeX Example: Project Report
%
% Source: http://www.howtotex.com
%
% Feel free to distribute this example, but please keep the referral
% to howtotex.com
% Date: March 2011 
% 
%%%%%%%%%%%%%%%%%%%%%%%%%%%%%%%%%%%%%%%%%%%%%%%%%%%%%%%%%%%%%%%%%%%%%%
% How to use writeLaTeX: 
%
% You edit the source code here on the left, and the preview on the
% right shows you the result within a few seconds.
%
% Bookmark this page and share the URL with your co-authors. They can
% edit at the same time!
%
% You can upload figures, bibliographies, custom classes and
% styles using the files menu.
%
% If you're new to LaTeX, the wikibook is a great place to start:
% http://en.wikibooks.org/wiki/LaTeX
%
%%%%%%%%%%%%%%%%%%%%%%%%%%%%%%%%%%%%%%%%%%%%%%%%%%%%%%%%%%%%%%%%%%%%%%
% Edit the title below to update the display in My Documents
%\title{Project Report}
%
%%% Preamble
\documentclass[paper=a4, fontsize=11pt abstract]{scrartcl}

%% Uncomment this line to use custom margins
\usepackage[top=2.5cm, bottom=2.5cm, left=2.5cm, right=2.5cm]{geometry}

\usepackage[T1]{fontenc}
\usepackage{fourier}

\usepackage[english]{babel}															% English language/hyphenation
\usepackage[protrusion=true,expansion=true]{microtype}	
\usepackage{amsmath,amsfonts,amsthm} % Math packages
\usepackage[pdftex]{graphicx}	
\usepackage{url}
\usepackage[titletoc,title]{appendix}
\usepackage[boxruled ]{algorithm2e}
\usepackage{listings}
\usepackage{bm}
\usepackage{gensymb}
\usepackage{float}
\usepackage{pythonhighlight}
\usepackage{csquotes}

\usepackage{biblatex}
\addbibresource{sample.bib}


%%% Custom sectioning
\usepackage{sectsty}
%\allsectionsfont{\centering \normalfont\scshape}
\allsectionsfont{\normalfont\scshape}


%%% Custom headers/footers (fancyhdr package)
\usepackage{fancyhdr}
\pagestyle{fancyplain}
\fancyhead{}											% No page header
\fancyfoot[L]{}											% Empty 
\fancyfoot[C]{}											% Empty
\fancyfoot[R]{\thepage}									% Pagenumbering
\renewcommand{\headrulewidth}{0pt}			% Remove header underlines
\renewcommand{\footrulewidth}{0pt}				% Remove footer underlines
\setlength{\headheight}{13.6pt}

%%% Equation and float numbering
\numberwithin{equation}{section}		% Equationnumbering: section.eq#
\numberwithin{figure}{section}			% Figurenumbering: section.fig#
\numberwithin{table}{section}				% Tablenumbering: section.tab#


%%% Maketitle metadata
\newcommand{\horrule}[1]{\rule{\linewidth}{#1}} 	% Horizontal rule

\title{
		%\vspace{-1in} 	
		\usefont{OT1}{bch}{b}{n}
		\normalfont \normalsize \textsc{Models for Business} \\ [25pt]
		\horrule{0.5pt} \\[0.4cm]
		\huge Estimating the Value of Ambition\\
		\horrule{2pt} \\[0.5cm]
}
\author{
		\normalfont
		\normalsize
        Robert deCarvalho\\[-3pt]
        \normalsize
        \today
}
\date{}



%%% Begin document
\begin{document}
\maketitle


\begin{abstract}
A paper in which we discuss and quantify the value that Ambition brings to your team.
\end{abstract}

%=============================================================================

\section{How Valuable is Ambition to My Team?}

At Ambition, we regularly get asked to estimate our ROI (Return on Investment).
Our customers want a way to put a dollar value on the benefits they receive from our product.
This question is almost impossible to answer in a one-size-fits-all kind of way.
Every team at every company is different.
The way they generate value is different.
The processes they follow are different.
Their workplace cultures are different.
You could even say that every team is it's own special snowflake.

Over time we have carefully observed our customers to gather insights into our generated value.
This paper presents an overview of what we have learned.
We feel very confident in some of these insights.  
Others are still a work in progress.
We will be careful to separate what we know from what we think is probably true.


We hope this analysis will help you maximize Ambition's potential.
We also hope that it becomes a source of general knowledge for maximizing your team's efficiency.

\section{The Ambition Tool Set}
We curated Ambition's tools to target elements of your team's collaboration.  We put a lot of thought into ensuring that each targeted element will drive increased productivity.
Furthermore -- and perhaps more importantly -- we were very careful to construct a system whose value is more than the sum of it's parts.
Below is a list of the tools in the Ambition product.
As you read through them, see if you can identify the common themes.

\begin{itemize}
    \item \textbf{Scoring}
    Scoring is the heart of Ambition's mission.
    Managers configure a system that delivers a single number (score) to their employees.
    This number communicates exactly what the manager expects from them and how well they are meeting those expectation.
    By having two scores that individually target ``Activities'' and ``Objectives'', employees not only understand what the manager wants, but the exact path they should take to get there.
    \textbf{Scoring drives transparency. Transparency fosters honesty.  Honesty builds trust.}
    
    \item \textbf{Leader-boards, Accolades, TVs, Chat-Integration}
    Ambition believes that happy employees are productive employees.
    We believe that praise, recognition, and honest feedback increase employee satisfaction.
    We also recognize managers sink a lot of time and effort into doing these things well.
    So we automated it.
    With Ambition, managers configure multiple tools that automate praise, recognition and feedback.
    \textbf{Employees are thirsty for praise, and hungry for a sense of purpose.
    Ambition automates giving it to them.}
    
    \item \textbf{Playing Games} At Ambition we believe that introducing ``play'' into people's jobs makes them more engaged.  We provide different games people can play at work where the victory goes to those who most effectively carry out the (manager-configurable) objective.  If we were to equate Ambition's different tools to elements of a good home-cooked meal, competitions would be the seasoning.  Thoughtfully used and presented with creativity, they are tremendously impactful. Applied haphazardly or overused they can destroy a perfectly good meal. \textbf{Competitions and games create excitement and drive engagement.}
    
    \item \textbf{Goals and Coaching} One of the most valuable features of Ambition is the way it automates the process of introducing real-time, configurable, performance-based metrics into manager-employee sessions.  In the sections below, we will take a deep-dive into the quantitative value generated by effective training and coaching.  It is essential that your team does this well. \textbf{Ambition automates real-time evidence-based training and coaching sessions.}
   
   \item \textbf{Workflows, Reports, Notifications}. Ambition provides a set of features that are focused on automating much of the mundane work required to effectively manage a team.  By creating custom-reports on Ambition-based metrics, automatically triggering notifications and/or action-items based on real time data Ambition significantly increases the managers' situational awareness (especially in remote-work environments) and frees-up valuable time for them to use doing what they do best -- inspiring and coaching. \textbf{Ambition enables vastly improved real-time situational awareness for mangers.  Especially in a remote-work context.} 
    
\end{itemize}

\section{The Structure of Ambition's Value}
In this section we will paint in broad strokes why we chose to implement the existing Ambition features and how they work together to significantly improve your team's production.  We deliberately keep this section non-technical leaving quantitative details for later sections designed to be consumed by those who relentlessly need to know and who are not afraid of math.

We will break-down Ambition's value into three overarching categories.
\begin{itemize}
    \item Performance lift (W.A.G. Value)
    \item Hiring/retention dynamics
    \item Automation
\end{itemize}
We will explain each category and outline how they are supported by Ambition's feature set.

\subsection{Performance Lift. (Better known as "The W.A.G. Value")}
As mentioned in the opening section, quantifying performance on a product with such a heavy people focus is not easy.  Although we have made good progress over the past few years, there are still some areas where we are confident that we provide value, but do not (yet) have the evidence to back it up.  \textbf{In this situation, we do what any self-respecting company would do. We just take a "Wild\ldots Guess."} Think of this section as a to-do list that we are working through to provide evermore compelling evidence of just how valuable Ambition is.
\begin{itemize}
    \item As a producer, having your colleagues, managers, and sometimes even VPs publicly heap praise on you when you knock it out of the park is a real incentive to do better.  \textbf{Your performance will go up.} 
    
    \item As a manager, going through the exercise of quantitatively specifying just how important each metric is to your employee's success forces a self-evaluation and clarity that is hard to achieve otherwise. \textbf{Creating formalized expectations focuses manager attention on what matters.  Performance will go up.}
    
    \item As a manager Ambition's automation capability (particularly with work-flows) enables processes that were simply not possible before. \textbf{Automation-enabled process changes will make performance go up.}
\end{itemize}

There are many other bullet points that could be itemized under the "W.A.G." flag, but we will stop here.  Let's just consider the three factors we mentioned.  You may not buy into all of them, but do you think it's plausible that taken together these W.A.G. elements could produce, say,  a $2\%$ boost in performance?  We do. We have anecdotal evidence from current customers that the lift can be much larger than this, but we don't want to strain our credibility. \textbf{So, just to be conservative, let's just say that a $2\%$ performance lift is about right.}

\subsection{Hiring/Retention Dynamics}
Hiring and retaining employees has a surprisingly large impact on your team's success.  We will go through a careful quantitative analysis of this in a later section, but let's just get a good understanding of why these dynamics are important and how Ambition helps drive them.

It all comes down to a very simple set of facts that pretty much everyone intuitively understands, but maybe hasn't spent the time quantifying.  There are basically two types of employees.  There are ``rookies'' that you are training up, and there are ``veterans'' that know what they are doing.  The veterans operate at a substantially higher level than the rookies do.  In an ideal world, your team would be comprised entirely of veterans.  The problem is, that we live in the real world, and people leave or get fired.  You will be continuously bombarded with the need to hire and train new employees.  This weighs on your team in two significant ways.  First, it takes time and money to recruit, hire, and train new employees.  That's just a raw expenditure of resources.  But in addition to that, your team production suffers because basically every rookie-occupied seat on your roster represents an opportunity cost due to the lower performance of that rookie.  You have got to minimize the time that employee stays in rookie status.  Your rookie-to-veteran conversion time needs to be as short as possible.

There are two important time scales that determine how your hiring and retention dynamics impact your productivity.  The first of these is how long you retain an employee.  The longer you keep employees, the more time you will have to recoup your training cost and the less resources you will have to devote to hiring and training.  The second of these is how long it takes to fully train up an employee.  Making this time as short as possible minimizes the resources expended in the training as well as the lost rookie-to-veteran opportunity cost.  As we will see in the quantitative sections below, hiring and retention dynamics impact your production through two simple metrics:
\begin{itemize}
    \item \textbf{The ratio of retention time to training time.}. This ratio controls how what fraction of your team is veterans and what fraction is rookies.  The higher this metric, the more veteran-like your team becomes.
    \item \textbf{The rookie production discount} If you can hire new employees and they are immediately as effective as the veterans, than training time kind of doesn't matter.  But the truth is that rookies are going to come in performing at an appreciably lower rate than the veterans.  This discounted production with respect to the veterans controls how different a team of veterans looks from a team of rookies.
\end{itemize}
If you look back at Ambition's feature set, you will see a common theme weaving its way through many of them.  Our coaching and goal capabilities are hyper-focused on reducing your training time. Our transparency and recognition features are focused on increasing retention time.  You can't really control the level of the talent pool you are hiring from (i.e. the rookie discount), and your veterans will naturally peak at some production value beyond which it's not worth your time coaching them.  The two things you can actually control is how long it takes to train a rookie, and how long your veterans stay with you.

In the quantitative section on hiring and retention we will develop a formula you can enter into a spreadsheet that will give you a dollar figure for the value you can expect Ambition to provide by improving your hiring/retention dynamics.

\textbf{Ambition has been designed to focus it's impact directly on the most important metric for driving value in the hiring/retention cycle. This value is directly quantifiable.} 


\subsection{Automation}
Ambition has been designed help reduce the day-to-day load of busy work that a manager needs to do in order to be effective.  Although this undoubtedly opens up new possibilities in terms of process, this is hard to quantify, so we simply focus on the time-saving this automation brings.  Using in-house surveys of Ambition users, we estimate that using Ambition saves them somewhere between 7 and 9 hours of busy-work per week.  Assuming a forty hour week this represents approximately a 20\% reduction in the time needed for a manager to to their job.  There are several ways you could try to calculate the impact of this free time, but we will go down the most simple obvious route.  \textbf{Ambition automation reduces your management overhead cost by approximately 20\%.}

\section{Qualitative Summary of Ambition Value}
Before we dive into the math-heavy quantitative portion of this paper, let's pause and take stock of what we have learned so far.

Ambition has been carefully designed address your most pressing management problems.  By reducing training time and increasing employee retention it specifically targets one of the more important employee management inefficiencies you will encounter.  By automating much of your manager's busy work it buys back valuable to time that can be put to more effective use.  Finally there are several benefits that we believe drive significant improvement in individual employee production.

We would like to close this non-quantitative portion of the paper by doing a remix of the food-based metaphor we first introduced in the first section.

\textbf{Getting the most out of your team is not so different than preparing a really good home-cooked meal. You have the raw ingredients (the employees), you have your appliances (your organizational structure, and you have your tools (whatever software you use to do your jobs).  Buying Ambition is like buying that high-end mixer, air-fryer or induction stove.  It gives you tools to not only make what you already do better, but also to create something you've never done before.}


\section{Quantitative Formulas for Ambition Value}
This section will focus on developing and presenting formulae for quantifying the different components of Ambitions value.  There will be some math involved, so we will summarize the results first, then present their derivations.  This will allow spreadsheet builders to quickly access the formulas they need without delving into the depths of the derivations.

\subsection{Definitions}
In order to think clearly about the valuation problem, we translate the various quantities into algebraic variables which are defined here.
\begin{itemize}
    \item $Q_0$\textbf{: The total value generated by the team (without Ambition)}  This is an estimate (in dollar terms) of the total annual value generated by this team prior to adopting Ambition.
    \item $Q_a$\textbf{: The total value generated by the team (with Ambition)}  This is an estimate (in dollar terms) of the total annual value generated by this team after adopting Ambition.
    \item $N_p$\textbf{: The total number of producers on the team.}  This is the number of people on the team that actually do the work.  It does not include managers or support staff.
    \item $N_m$\textbf{: The total number of managers on the team.}  This is the number of managers on the team.
    \item $N_r$\textbf{: The number or ``rookies'' on the team.}  This is the number of producers that are not yet fully trained.  They still require significant coaching/experience to improve their performance. This number will fluctuate as people are hired/fired, but we will compute a rough average for this number in the equations below.
    \item $N_v$\textbf{: The number or ``veterans'' on the team.}  This is the number of fully-trained producers on the team. They are expected to achieve full-performance. This number will fluctuate as people are hired/fired, but we will compute a rough average for this number in the equations below.
    \item $q_{v0}$\textbf{: The pre-ambition annual value generated by a ``veteran.''}  This is the amount of value we expect from a fully-trained producer.  This number will, of course, vary from person to person, but we will estimate it in the equations below as the proportion of the team's total value generated by veterans divided by the average number of veterans.
    \item $\alpha$\textbf{: The ``rookie'' discount rate.}  This is the fraction by which we need to reduce the expected value generated by the rookies.  Until they are fully trained, rookies are assumed to be ``worth'' this fraction of a veteran in terms of dollars value generated.
    \item $T_c$\textbf{: Time to churn} This is the average length of time that a producer remains employed at the company.
    \item $\beta$\textbf{: Overall performance lift due to Ambition.} This is just the factor by which you multiply total team production to capture the holistic improvements that Ambition generates. 
    \item $T_t$\textbf{: Time to train} This is the average length of time it takes to fully ramp a producer, transforming them from a ``rookie'' to a ``veteran.'' 
    \item $\gamma$\textbf{: The churn-to-train ratio prior to adopting Ambition} This is the ratio of the average time a producer stays with the team compared to how long it takes to train them.  You really want to maximize this number, otherwise you'll be spending way too many resources training talent for your competitors.
    \item $\gamma_a$\textbf{: The churn-to-train ratio after adopting Ambition} This is the resulting time ratio after Ambition shortens training time and lengthens churn time. 
    \item $\chi_L$\textbf{: The fractional change in value driven by holistic performance lift} This is the fraction by which an individual producers' performance is lifted due to Ambition's holistic impact.
    \item $\Delta_L$\textbf{: The dollar change in value driven by holistic performance lift} This is the dollar impact that Ambition's holistic performance lift enables for the team.
    \item $\chi_a$\textbf{: Automation time factor} Engaging Ambition's automation capability reduces a managers required work hours to this fraction of what it was prior to Ambition
    \item $c_m$\textbf{: The total loaded cost of a single manager} This is how much it costs the company to keep a single manager on staff.
    \item $c_h$\textbf{: Hiring/recruiting cost for one producer} This is how much it costs to recruit and hire a single producer.
    \item $\Delta_a$\textbf{: The dollar change in value driven by automation} This is the dollar impact of Ambition's automation.
\end{itemize}

\subsection{Performance Lift Valuation}
This is perhaps the easiest of all valuations to perform, so we tackle it first.  If we assume each producer elevates their production by a factor of $\chi_L$, than the entire team will elevate it's performance by the same ratio and we obtain
\begin{equation}
    \Delta_L = Q_0\left(\chi_L - 1\right)
\end{equation}
\subsection{Automation Value}
It is also straightforward to compute the value created by Ambition's automation tools. The idea here is that we make managers more efficient by automating busy work.  This enables them to manage a larger team thereby increasing your manager to employee ratio.  To quantify this, we simply take the fraction of time we saved the manager and apply that same fraction to their loaded salary cost.  We do this for all managers and arrive at the following expression for dollar value generated by Ambition automation.
\begin{equation}
    \Delta_a = N_m c_m \left(1 - \chi_a\right)
\end{equation}
\subsection{Hiring and Retention Dynamics}
Computing the dollar value of hiring and retention dynamics is the most involved so we have saved it for last.  We start out by simply stating the result and pushing the derivation into subsections. In all these equations, the meaning of each symbol is given in the definitions section.

The first thing we care about is estimating \textbf{the value generated by a single veteran producer prior to adopting ambition.}

\begin{equation}
    q_{v0} = \frac{Q_0}{N_p}\left(\frac{1 + \gamma}{\alpha + \gamma}\right).
\end{equation}

We can then use the veteran production value to obtain two equivalent expressions for \textbf{the total team production value after adopting Ambition.}
\begin{align}
    Q &= \chi_L q_{v0} N_p\frac{\left(\frac{\alpha}{\chi_L}\right) + \gamma_a}{1 + \gamma_a} \label{Q0} \\
    Q &= \chi_L Q_0  \left(\frac{1+\gamma}{1 + \gamma_a}\right) \left(\frac{\frac{\alpha}{\chi_L} + \gamma_a  }{\alpha + \gamma}\right).
\end{align}

\subsubsection{Model Setup}
We will make the following assumptions about your team.  It consists of $N_p$ producers being managed by $N_m$ managers.  The average producer stays with your company for a duration of $T_c$ (we call this time-to-churn).  It takes a time $T_t$ (time-to-train) to fully ramp a new hire transforming them from a rookie to a veteran. The total annual value generated by your team prior to adopting Ambition is $Q_0$, and the fully loaded cost of a manger is $c_m$.  When people churn, it costs $c_h$ to recruit and hire their replacement.

\subsubsection{Computing time-to-churn}
You may not immediately know how long the average employee stays with the company, so this section gives you a quick way to estimate it given numbers you probably know off the top of your head.

Let's say that on a team of $N_p$ producers, you churn $N_c$ of them in some period, $\tau$.  So, for example, you could churn 6 producers a year on your team of 20 people.  In this case, $N_p=20$, $N_c=6$ and $\tau=1\text{ year}$. If we assume that the churn events are completely random, this becomes a classic problem in survival analysis -- that of a constant hazard function. All this means is that we are justified in estimating the time-to-churn using the following formula
\begin{equation}
    T_c = \tau \frac{N_p}{N_c}.
\end{equation}
So, for our example,
\begin{align}
    T_c &= \left(\text{1 year}\right)\frac{20}{6} \\
    T_c &= 3.33\text{ years}.
\end{align}

\subsubsection{Computing Rookie/Veteran Proportions}
Intuitively we know that our team will be composed mostly of veterans when our churn rate is really low.  Similarly, if our churn rate is really high, we know that our team is going to be mostly composed of rookies.  Let's put some effort into making this more quantitative.

Let's think about how people transition through the system.  The rate at which producers churn from the team is
\begin{equation}
    R_c = \frac{N_p}{T_c}
\end{equation}
If our team is to remain at the same size, we need to hire at the same rate as people churn so that means our hiring rate must be equal to our churn rate, or
\begin{equation}
    R_h = R_c = \frac{N_p}{T_c}.
\end{equation}

Now, we said that it takes some time, $T_t$ to fully ramp producers from rookies to veterans.  That means that the rate at which we lose rookies because they've been transformed to veterans is
\begin{equation}
    R_t = \frac{N_r}{T_t}.
\end{equation}
Furthermore, we can separately compute the rate at which rookies and veterans churn
\begin{align}
    R_r &= \frac{N_r}{T_c} \\
    R_v &= \frac{N_v}{T_c}.
\end{align}

Now lets put all this together.  If our team's composition is not changing, this means that the rate at which we hire people must be balanced by the the rate at which we churn rookies and the rate at which we transform them into veterans, or
\begin{align}
    R_h &= R_r + R_t \\
    \frac{N_p}{T_c} &= \frac{N_r}{T_c} + \frac{N_r}{T_t} \\
    N_p &= N_r\left(1 + \frac{T_c}{T_t}\right) \\
    N_r &= \frac{N_p}{\left(1 + \frac{T_c}{T_t}\right)} = \frac{N_p}{\left(1 + \gamma\right)} \label{Nr}
\end{align}
Similarly, the rate at which we churn veterans must be equal to the rate at which create them if our team has a stable composition.  So
\begin{align}
    R_t &= R_v \\
    \frac{N_r}{T_t} &= \frac{N_v}{T_c} \\
    N_v &= \frac{T_c}{T_t}N_r = \gamma N_r \\
    N_v &= \frac{\gamma N_p}{\left(1 + \gamma\right)} \label{Nv}
\end{align}

Equations \ref{Nr} and \ref{Nv} are the stable proportions of veterans to rookies given the ratio of churn to train time.

\subsubsection{Computing Value as a function of Churn-to-Train Time Ratio}
Recall that we assumed that rookies generated value only at a rate of $\alpha$ compared to our veterans.  This means, that if your team's total production prior to adopting Ambition was $Q_0$, the breakdown of how that came from rookies and veterans must have been
\begin{equation}
    Q_0 = \alpha q_{v0} N_r + q_{v0} N_v = q_{v0}\left(\alpha N_r + N_v\right).
\end{equation}
Substituting our previously obtained expressions for the rookie/veteran populates, we obtain
\begin{align}
    Q_0 &= q_{v0}\left(\frac{\alpha N_p}{1 + \gamma} + \frac{\gamma N_p}{1 + \gamma}\right)  \\
    Q_0 &= q_{v0} N_p\frac{\alpha + \gamma}{1 + \gamma} \label{Q0}.
\end{align}
Let's take a look at this equation because it's interesting.  think about what happens when your churn time is much much longer than your train time.  In this case $\gamma >> 1$ and your total production will approach $q_{v0}N_p$. This would be like treating every producer like a veteran.  This makes sense because most of your producers WILL be veterans under these circumstances.  Now consider the case where your churn time is much smaller than your train time.  This would reflect the case where you lose people before you can fully train them.  In this case $\gamma << 1$ and your total production will approach $\alpha q_{v0} N_p$.  This the production you would expect if you had a team full of rookies all having reduced production.

We can obtain one more useful relation if we invert Equation (\ref{Q0}).  Solving for $q_{v0}$ gives us an expression for the veteran-generated revenue prior to Ambition being adopted,
\begin{equation}
    q_{v0} = \frac{Q_0}{N_p}\left(\frac{1 + \gamma}{\alpha + \gamma}\right).
\end{equation}

Now to pull it all together by thinking about how Ambition impacts each variable in the equation.  Ambition will not impact the skill of incoming rookies, but it will elevate the performance of veterans by a factor of $\chi_L$.  This means that the rookies will have a steeper hill to climb, and post ambition their performance reduction will actually be $\alpha / \chi_L$.  Furthermore the churn-to-train time ratio will go from $\gamma$ to $\gamma_a$.  Making these three substitutions in Equation (\ref{Q0}), we obtain an expression for the team production after adopting Ambition.
\begin{align}
    Q &= \chi_L q_{v0} N_p \frac{\left(\frac{\alpha}{\chi_L}\right) + \gamma_a}{1 + \gamma_a} \\
    Q &= \chi_L \left(     \frac{Q_0}{N_p}\left(\frac{1 + \gamma}{\alpha + \gamma}\right)             \right) N_p \frac{\left(\frac{\alpha}{\chi_L}\right) + \gamma_a}{1 + \gamma_a} \\
    Q &= \chi_L Q_0  \left(\frac{1 + \gamma}{\alpha + \gamma}  \right)  \frac{\left(\frac{\alpha}{\chi_L}\right) + \gamma_a}{1 + \gamma_a} \\
    Q &= \chi_L Q_0  \left(\frac{1+\gamma}{1 + \gamma_a}\right) \left(\frac{\frac{\alpha}{\chi_L} + \gamma_a  }{\alpha + \gamma}\right)
\end{align}

% \printbibliography

%%% End document
\end{document}