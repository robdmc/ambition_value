%%%%%%%%%%%%%%%%%%%%%%%%%%%%%%%%%%%%%%%%%%%%%%%%%%%%%%%%%%%%%%%%%%%%%%
% LaTeX Example: Project Report
%
% Source: http://www.howtotex.com
%
% Feel free to distribute this example, but please keep the referral
% to howtotex.com
% Date: March 2011 
% 
%%%%%%%%%%%%%%%%%%%%%%%%%%%%%%%%%%%%%%%%%%%%%%%%%%%%%%%%%%%%%%%%%%%%%%
% How to use writeLaTeX: 
%
% You edit the source code here on the left, and the preview on the
% right shows you the result within a few seconds.
%
% Bookmark this page and share the URL with your co-authors. They can
% edit at the same time!
%
% You can upload figures, bibliographies, custom classes and
% styles using the files menu.
%
% If you're new to LaTeX, the wikibook is a great place to start:
% http://en.wikibooks.org/wiki/LaTeX
%
%%%%%%%%%%%%%%%%%%%%%%%%%%%%%%%%%%%%%%%%%%%%%%%%%%%%%%%%%%%%%%%%%%%%%%
% Edit the title below to update the display in My Documents
%\title{Project Report}
%
%%% Preamble
\documentclass[paper=a4, fontsize=11pt abstract]{scrartcl}

%% Uncomment this line to use custom margins
\usepackage[top=2.5cm, bottom=2.5cm, left=2.5cm, right=2.5cm]{geometry}

\usepackage[T1]{fontenc}
\usepackage{fourier}

\usepackage[english]{babel}															% English language/hyphenation
\usepackage[protrusion=true,expansion=true]{microtype}	
\usepackage{amsmath,amsfonts,amsthm} % Math packages
\usepackage[pdftex]{graphicx}	
\usepackage{url}
\usepackage[titletoc,title]{appendix}
\usepackage[boxruled ]{algorithm2e}
\usepackage{listings}
\usepackage{bm}
\usepackage{gensymb}
\usepackage{float}
\usepackage{pythonhighlight}
\usepackage{csquotes}

\usepackage{biblatex}
\addbibresource{sample.bib}


%%% Custom sectioning
\usepackage{sectsty}
%\allsectionsfont{\centering \normalfont\scshape}
\allsectionsfont{\normalfont\scshape}


%%% Custom headers/footers (fancyhdr package)
\usepackage{fancyhdr}
\pagestyle{fancyplain}
\fancyhead{}											% No page header
\fancyfoot[L]{}											% Empty 
\fancyfoot[C]{}											% Empty
\fancyfoot[R]{\thepage}									% Pagenumbering
\renewcommand{\headrulewidth}{0pt}			% Remove header underlines
\renewcommand{\footrulewidth}{0pt}				% Remove footer underlines
\setlength{\headheight}{13.6pt}

%%% Equation and float numbering
\numberwithin{equation}{section}		% Equationnumbering: section.eq#
\numberwithin{figure}{section}			% Figurenumbering: section.fig#
\numberwithin{table}{section}				% Tablenumbering: section.tab#


%%% Maketitle metadata
\newcommand{\horrule}[1]{\rule{\linewidth}{#1}} 	% Horizontal rule

\title{
		%\vspace{-1in} 	
		\usefont{OT1}{bch}{b}{n}
		\normalfont \normalsize \textsc{Models for Business} \\ [25pt]
		\horrule{0.5pt} \\[0.4cm]
		\huge Estimating the Value of Ambition\\
		\horrule{2pt} \\[0.5cm]
}
\author{
		\normalfont
		\normalsize
        Robert deCarvalho\\[-3pt]
        \normalsize
        \today
}
\date{}



%%% Begin document
\begin{document}
\maketitle


\begin{abstract}
A paper in which we discuss and quantify the value that Ambition brings to your team.
\end{abstract}

%=============================================================================

\section{How Valuable is Ambition to My Team?}
At Ambition, we regularly get asked how to quantify the value we provide to our customers.  This question is almost impossible to answer in a one-size-fits-all kind of way because every team at every company is different.  The way they generate value is different.  The processes they follow are different. The workplace culture is different.  One might even say that every team is it's own special snowflake.

Ambition's tools are targeted at streamlining the way teams work together to drive value. If your team is mature and clicking along at a healthy pace, Ambition's value to you will be different from another team that is still evolving on how everyone works together, how people are trained, and how to determine ``what good looks like.''

Despite the vast differences that exist between teams that currently use Ambition, we have, over time, developed a reliable guide for estimating -- in actual dollar terms -- the value that Ambition brings to your team.  This paper is an in-depth guide to the way we think about value. We hope that it will serve not only as a guide for your team to implement Ambition, but also as a source of general knowledge on how to maximize team efficiency in ways you may not have previously considered.


\section{The Ambition Tool Set}
Each of Ambition's tools has been refined over time to be laser focused on a specific element of your team's dynamics and process.  We strategically chose these elements to harness human nature in a way that quantifiably drives productivity.  Although you may already have a solid grasp on the different tools Ambition offers, we present a quick overview here so that we can explain in a later section how we weave them together to maximize impact.
\begin{itemize}
    \item \textbf{Scoring} Scoring lies at the heart of Ambition's mission.  We allow you to create a single number that efficiently communicates to your employees exactly what is expected of them.  Given the ability to score ``Activities'' and ``Objectives'' differently, there is perhaps no better way to communicate to your team (in an automated way) exactly what you expect from them, and what they should be doing to meet those expectations.  \textbf{Scoring drives transparency. Transparency fosters honesty.  Honesty builds trust.}
    
    \item \textbf{Leader-boards, Accolades, TVs, Chat-integration} At Ambition we believe that happy employees are productive employees.  We believe that praise, recognition, and real-time, honest feedback are essential to driving employee satisfaction.  We have created several different channels whereby managers can automate the process of making employees feel great about their accomplishments. \textbf{Employees are thirsty for praise and recognition, and they are hungry for a sense of purpose.  Ambition automates the process of satiating them.}
    
    \item \textbf{Playing Games} At Ambition we believe that introducing ``play'' into people's jobs makes them more engaged.  We provide different games people can play at work where the victory goes to those who most effectively carry out the (manager-configurable) objective.  If we were to equate Ambition's different tools to elements of a good home-cooked meal, competitions would be the seasoning.  Thoughtfully used and presented with creativity, they are tremendously impactful. Applied haphazardly or overused they can destroy a perfectly good meal. \textbf{Competitions and games create excitement and drive engagement.}
    
    \item \textbf{Goals and Coaching} One of the most valuable features of Ambition is the way it automates the process of introducing real-time, configurable, performance-based metrics into manager-employee sessions.  In the sections below, we will take a deep-dive into the quantitative value generated by effective training and coaching.  It is essential that your team does this well. \textbf{Ambition automates real-time evidence-based training and coaching sessions.}
   
   \item \textbf{Workflows, Reports, Notifications}. Ambition provides a set of features that are focused on automating much of the mundane work required to effectively manage a team.  By creating custom-reports on Ambition-based metrics, automatically triggering notifications and/or action-items based on real time data Ambition significantly increases the managers' situational awareness (especially in remote-work environments) and frees-up valuable time for them to use doing what they do best -- inspiring and coaching. \textbf{Ambition enables vastly improved real-time situational awareness for mangers.  Especially in a remote-work context.} 
    
\end{itemize}

\section{The Structure of Ambition's Value}
In this section we will paint in broad strokes why we chose to implement the existing Ambition features and how they work together to significantly improve your team's production.  We deliberately keep this section non-technical leaving quantitative details for later sections designed to be consumed by those who relentlessly need to know and who are not afraid of math.

We will break-down Ambition's value into three overarching categories.
\begin{itemize}
    \item Performance lift (W.A.G. Value)
    \item Hiring/retention dynamics
    \item Automation
\end{itemize}
We will explain each category and outline how they are supported by Ambition's feature set.

\subsection{Performance Lift. (Better known as "The W.A.G. Value")}
As mentioned in the opening section, quantifying performance on a product with such a heavy people focus is not easy.  Although we have made good progress over the past few years, there are still some areas where we are confident that we provide value, but do not (yet) have the evidence to back it up.  \textbf{In this situation, we do what any self-respecting company would do. We just take a "Wild\ldots Guess."} Think of this section as a to-do list that we are working through to provide evermore compelling evidence of just how valuable Ambition is.
\begin{itemize}
    \item As a producer, having your colleagues, managers, and sometimes even VPs publicly heap praise on you when you knock it out of the park is a real incentive to do better.  \textbf{Your performance will go up.} 
    
    \item As a manager, going through the exercise of quantitatively specifying just how important each metric is to your employee's success forces a self-evaluation and clarity that is hard to achieve otherwise. \textbf{Creating formalized expectations focuses manager attention on what matters.  Performance will go up.}
    
    \item As a manager Ambition's automation capability (particularly with work-flows) enables processes that were simply not possible before. \textbf{Automation-enabled process changes will make performance go up.}
\end{itemize}

There are many other bullet points that could be itemized under the "W.A.G." flag, but we will stop here.  Let's just consider the three factors we mentioned.  You may not buy into all of them, but do you think it's plausible that taken together these W.A.G. elements could produce, say,  a $2\%$ boost in performance?  We do. We have anecdotal evidence from current customers that the lift can be much larger than this, but we don't want to strain our credibility. \textbf{So, just to be conservative, let's just say that a $2\%$ performance lift is about right.}

\subsection{Hiring/Retention Dynamics}
Hiring and retaining employees has a surprisingly large impact on your team's success.  We will go through a careful quantitative analysis of this in a later section, but let's just get a good understanding of why these dynamics are important and how Ambition helps drive them.

It all comes down to a very simple set of facts that pretty much everyone intuitively understands, but maybe hasn't spent the time quantifying.  There are basically two types of employees.  There are ``rookies'' that you are training up, and there are ``veterans'' that know what they are doing.  The veterans operate at a substantially higher level than the rookies do.  In an ideal world, your team would be comprised entirely of veterans.  The problem is, that we live in the real world, and people leave or get fired.  You will be continuously bombarded with the need to hire and train new employees.  This weighs on your team in two significant ways.  First, it takes time and money to recruit, hire, and train new employees.  That's just a raw expenditure of resources.  But in addition to that, your team production suffers because basically every rookie-occupied seat on your roster represents an opportunity cost due to the lower performance of that rookie.  You have got to minimize the time that employee stays in rookie status.  Your rookie-to-veteran conversion time needs to be as short as possible.

There are two important time scales that determine how your hiring and retention dynamics impact your productivity.  The first of these is how long you retain an employee.  The longer you keep employees, the more time you will have to recoup your training cost and the less resources you will have to devote to hiring and training.  The second of these is how long it takes to fully train up an employee.  Making this time as short as possible minimizes the resources expended in the training as well as the lost rookie-to-veteran opportunity cost.  As we will see in the quantitative sections below, hiring and retention dynamics impact your production through two simple metrics:
\begin{itemize}
    \item \textbf{The ratio of retention time to training time.}. This ratio controls how what fraction of your team is veterans and what fraction is rookies.  The higher this metric, the more veteran-like your team becomes.
    \item \textbf{The rookie production discount} If you can hire new employees and they are immediately as effective as the veterans, than training time kind of doesn't matter.  But the truth is that rookies are going to come in performing at an appreciably lower rate than the veterans.  This discounted production with respect to the veterans controls how different a team of veterans looks from a team of rookies.
\end{itemize}
If you look back at Ambition's feature set, you will see a common theme weaving its way through many of them.  Our coaching and goal capabilities are hyper-focused on reducing your training time. Our transparency and recognition features are focused on increasing retention time.  You can't really control the level of the talent pool you are hiring from (i.e. the rookie discount), and your veterans will naturally peak at some production value beyond which it's not worth your time coaching them.  The two things you can actually control is how long it takes to train a rookie, and how long your veterans stay with you.

In the quantitative section on hiring and retention we will develop a formula you can enter into a spreadsheet that will give you a dollar figure for the value you can expect Ambition to provide by improving your hiring/retention dynamics.

\textbf{Ambition has been designed to focus it's impact directly on the most important metric for driving value in the hiring/retention cycle. This value is directly quantifiable.} 


\subsection{Automation}
Ambition has been designed help reduce the day-to-day load of busy work that a manager needs to do in order to be effective.  Although this undoubtedly opens up new possibilities in terms of process, this is hard to quantify, so we simply focus on the time-saving this automation brings.  Using in-house surveys of Ambition users, we estimate that using Ambition saves them somewhere between 7 and 9 hours of busy-work per week.  Assuming a forty hour week this represents approximately a 20\% reduction in the time needed for a manager to to their job.  There are several ways you could try to calculate the impact of this free time, but we will go down the most simple obvious route.  \textbf{Ambition automation reduces your management overhead cost by approximately 20\%.}

\section{Qualitative Summary of Ambition Value}
Before we dive into the math-heavy quantitative portion of this paper, let's pause and take stock of what we have learned so far.

Ambition has been carefully designed address your most pressing management problems.  By reducing training time and increasing employee retention it specifically targets one of the more important employee management inefficiencies you will encounter.  By automating much of your manager's busy work it buys back valuable to time that can be put to more effective use.  Finally there are several benefits that we believe drive significant improvement in individual employee production.

We would like to close this non-quantitative portion of the paper by doing a remix of the food-based metaphor we first introduced in the first section.

\textbf{Getting the most out of your team is not so different than preparing a really good home-cooked meal. You have the raw ingredients (the employees), you have your appliances (your organizational structure, and you have your tools (whatever software you use to do your jobs).  Buying Ambition is like buying that high-end mixer, air-fryer or induction stove.  It gives you tools to not only make what you already do better, but also to create something you've never done before.}


\section{OLD OLD OLD OLD The Components of Ambition's Value}
For the purposes of this paper, I will assume that you are evaluating Ambition for deployment to a specific team.  I assume that you have a good estimate for the annual dollar value this team is responsible for producing.  I will assume you know a bit about the cost structure of the team as well as the hiring/termination dynamics of the team.

We will break down Ambition's value into four components:
\begin{itemize}
    \item \textbf{Performance Lift:} Ambition provides tooling to automate much of the busy-work that bogs down managers.  Producers automatically receive real-time performance feedback motivating them to tweak their activity on the fly. This real time feedback, combined with the social rewards of public recognition and the desire to compete, drives up their overall production.
    \item \textbf{Improved Training Efficiency:} Transactional jobs (e.g. sales-reps, call-center employees, etc) are notorious for suffering high employee churn.  The cost associated with hiring and training new employees, combined with their decreased efficiency during training, induces a significant drag on the team's production. Ambitions tools automate the tedious and time-intensive bookkeeping required to conduct training in a real-time performance-based environment. This drives down the time-to-full-value for new employees and increases manager capacity.
    \item \textbf{Improved Team Culture} The longer a team can retain the talent the higher returns they will realize from their training investment.  Ambition's tools automate much of the work required to give producers clear expectations, communicate their real-time performance against those expectations and reward them with highly visible public recognition for good work. The resulting cultural improvement lengthen the average time a producer stays on the team.
    \item \textbf{Increased Manager Capacity} Ambition centralizes the information managers need to do their work, and automates much of the mundane tasks managers need to be effective.  This opens up capacity for them to more effectively manage larger teams. 
\end{itemize}
The remainder of this paper will be devoted to trying to put a dollar value on each of these four components.  Different teams are all unique, and one-size definitely does not fit all, but we can come up with generalizations that lead to a reasonable estimate of the value Ambition generates.

\section{Definitions}
In this section I will define algebraic variables pertaining to relevant business metrics. I will than manipulate these variables to arrive a formulae that can be entered into spreadsheets for customizing value computations to a particular business.  In these definitions.
\begin{itemize}
    \item $Q$\textbf{: The total value generated by the team}  This is an estimate (in dollar terms) of the total annual value generated by this team.
    \item $N_p$\textbf{: The total number of producers on the team.}  This is the number of people on the team that actually do the work.  It does not include managers or support staff.
    \item $N_r$\textbf{: The number or ``rookies'' on the team.}  This is the number of producers that are not yet fully trained.  They still require significant coaching/experience to improve their performance. This number will fluctuate as people are hired/fired, but we will compute a rough average for this number in the equations below.
    \item $N_v$\textbf{: The number or ``veterans'' on the team.}  This is the number of fully-trained producers on the team. They are expected to achieve full-performance. This number will fluctuate as people are hired/fired, but we will compute a rough average for this number in the equations below.
    \item $q_v$\textbf{: The annual value generated by a ``veteran.''}  This is the amount of value we expect from a fully-trained producer.  This number will, of course, vary from person to person, but we will estimate it in the equations below as the proportion of the team's total value generated by veterans divided by the average number of veterans.
    \item $\alpha$\textbf{: The ``rookie'' discount rate.}  This is the fraction by which we need to reduce the expected value generated by the rookies.  Until they are fully trained, rookies are assumed to be ``worth'' this fraction of a veteran in terms of dollars value generated.
    \item $T_c$\textbf{: Time to churn} This is the average length of time that a producer remains employed at the company.
    \item $T_t$\textbf{: Time to train} This is the average length of time it takes to fully ramp a producer, transforming them from a ``rookie'' to a ``veteran.'' 
    \item $\gamma$\textbf{: The churn-to-train ratio} This is the ratio of the average time a producer stays with the team compared to how long it takes to train them.  You really want to maximize this number, otherwise you'll be spending way too many resources training talent for your competitors.
\end{itemize}

\section{Hiring and Retention Dynamics}
For the transactional-type jobs that Ambition has been designed for, employee churn is a significant consideration.  This section is devoted to creating a simple model that captures the dynamics of a typical team as they hire, train, and churn employees.

\subsection{Population Dynamics}
Let's consider a team with budget for $N_p$ producers.  The leaders of this team are responsible for maximizing the value generated by that team while being constrained to have no more than $N_p$ producers on the roster.  In an ideal world, the team would consist entirely of fully-trained, highly-effective producers that each perform at the expected level.  In the real world, people move on to greener pastures, and the leaders are forced to hire untrained replacements.  There are two timescales that come into play here.
\begin{itemize}
    \item How long are you able to hold on to your employees?
    \item How long does it take to fully-ramp new employees?
\end{itemize}
It's clear that you want to maximize how long you retain employees and minimize how long it takes to train them. It should be pretty obvious that if people leave you before you get them trained, not only are you bearing the expense of the failed training, but you also suffer from their reduced, untrained performance while they were with you.  This brings us to  the most important metric to consider when thinking about employee dynamics.  You really need to maximize the ratio of churn time to train time, which we will call gamma ($\equiv$ means ``defined to be'')

\begin{equation}
    \gamma \equiv \frac{T_c}{T_t}
\end{equation}

Appendix XXX goes through the math to show two equations you can put in your spreadsheet for how many ``rookies'' and how many ``veterans'' you should expect to have on your team.  It turns out that the proportion of veterans to rookies is governed by $\gamma$, the ratio of churn time to train time.
\begin{align}
    N_r &= \frac{N_p}{1 + \gamma} \\
    N_v &= \frac{\gamma N_p}{1 + \gamma},
\end{align}
where, $N_p$ is the total number of producers on your team, $N_r$ is the expected number of rookies and $N_v$ is the expected number of veterans.  So here's the somewhat obvious punchline.

\textbf{The higher you can get your ratio of churn-time to train-time, the more veterans you will have on your team.}

\subsection{Production Dynamics}
Now, we can use the population dynamics above to come up with an important result.  All the math is shown in appendix XXX, but here is the main result.

\begin{equation}
    Q = q_v N_p \frac{\alpha + \gamma}{1 + \gamma},
\end{equation}
where $Q$ is your team's production, $q_v$ is the average production for a veteran, $N_p$ is the total number of producers you have, $\gamma$ is the churn-to-train ratio and $\alpha$ is the reduction factor for rookie production.  Here's the punchline.

\textbf{The higher you can get your churn-time to train-time ratio, the more your team will look like veterans at full production and the less your team will look like rookies with reduced production.}

\subsection{How Ambition Fills your Team With Better People}
In this section we demonstrated how crucial it is that you to retain people for much much longer than it takes you to train them.  You need to maximize employee retention time and minimize training time.  Ambition provides two tools that are hyper-focused on driving precisely these two outcomes.
\subsubsection{Automated Coaching}
Ambition's coaching module is unlike any other in that it automates the process of bringing in real-time performance metrics into coaching sessions.  Your managers can spend far less time preparing for their coaching sessions, and much more time where they provide most value: having meaningful coaching conversations. 

\textbf{Ambition's coaching module shortens the amount of time it takes to transform rookies into veterans. It reduces your training time.}

\subsubsection{Automated Feedback and Public Recognition}
With it's highly configurable scoring system, Ambition simplifies the process of giving your employees a simple number to understand how they are performing relative to your expectations.  With it's numerous avenues for heaping praise, Ambition automates the process of your employees feeling appreciated and getting publicly recognized for their good work.
\textbf{Ambition's public-recognition tools automate the process of doling out praise and appreciation.  A positively affirmed team is a happier team.  Happy teams stay together longer.  Ambition increases your employee retention time.}

\subsection{Hiring/Retention Summary}
Your team is constrained by a roster-limiting budget.  There's also a point of diminishing returns after which additional training for your team will not significantly increase production.  Basically, your veterans will attain some benchmark level of annual production per person, and there's not a lot more you can do about that.  The easiest remaining lever you can control is how quickly you ramp new employees and how long you retain the veterans.

Ambition has tools directly aimed at reducing training time and increasing churn time.  Equation (XX) can be put in a spreadsheet to directly calculate the value this will have on your production.

\section{Building a Spreadsheet to Evaluate Ambition}
\begin{itemize}
    \item Compute value of employee-dynamics shift
    \begin{itemize}
        \item Compute the status-quo production-per-veteran
        \item Estimate Ambition's impact on churn time and training time
        \item Estimate training time percentage value
        \item Estimate churn time percentage value
        \item Estimate combination of churn and train time improvements
    \end{itemize}
    \item Compute time savings of automating reporting-data gathering
    \begin{itemize}
        \item Assume fractional time savings get applied to scaling team.
        \item Assume team production scales proportionally to roster size.
    \end{itemize}
    \item Estimate performance ``lift'' driven by recognition/competition/coaching (POMA)
\end{itemize}







\section{Move this to appendix}
Recall that one of our assumptions was that rookies only produce some fraction, $\alpha$, of what veterans produce.  Let's apportion the team's total production, $Q$ into what is produced by rookies, $Q_r$, and what is produced by veterans $Q_v$,

\begin{align}
    Q &= Q_v + Q_r \\
    Q &= q_v N_v + \alpha q_v N_r \\ 
    Q &= q_v \frac{\gamma N_p}{1 + \gamma} + \alpha q_v \frac{\gamma N_p}{1 + \gamma} \\
    Q &= q_v N_p \frac{\alpha + \gamma}{1 + \gamma}
\end{align}

Now rearrange this to figure out $q_v$ for status-quo.

\begin{equation}
    q_v = \frac{Q}{N_p}\frac{1+ \gamma}{\alpha + \gamma}
\end{equation}




\section{Deriving Recurring Revenue Equation}
We are considering the total value of a contract over a time horizon of $N$ contract periods.  This value quantifies in a revenue stream corresponding to all the payments we expect to receive over that time period, and is given by
\begin{equation}
    V = \sum_{k=0}^{m_0 M N - 1} r_k,
\end{equation}
where $k$ indexes the payments,  $m_0$ is the number of payments in a year, $m_0 M$ is the number of payments in an M-year contract, $m_0 m N$ is the number of payments over the entire time interval we care about, and $r_k$ is the discounted value of some payment we will receive in the future.

Let's talk more about this discounting. We have already stated that we will be using an annual discount rate of $i_0$ to quantify the time value of money.  Since we are evenly distributing our revenue over $m_0$ payments in a given year, the interest rate must be prorated according to this billing period to give.
\begin{equation}
    i = \frac{i_0}{m_0},
\end{equation}
the per-period interest rate.  What this says is that I am indifferent between receiving a payment of $r_0$ at the beginning of the period or a payment of $(1 + i) r_0$ at the end of the period. Said another way, getting money now is  $(1 + i)$ times more valuable then getting it later, so 
\begin{align}
    r_\textrm{now} &= (1 + i) r_\textrm{later} \\
    r_\textrm{later} &= \frac{1}{(1 + i)} r_\textrm{now}.
\end{align}
This means, that if I receive a payment $k$ billing periods into the future, I should discount it's value by
\begin{align}
    r_\textrm{discounted} &= \frac{1}{(1 + i)^k} r_0 \\
    r_\textrm{discounted} &= \beta^k r_0.
\end{align}
where
\begin{equation}
    \beta \equiv \frac{1}{1 + \frac{i_0}{m_0}}.
\end{equation}

But now let's think about those payment values, $r_0$.  We are guaranteed those payments as long as the current contract is in force, but as soon as it expires, we have no guarantee that the customer will renew.  The right thing to do here, is to assume that they will renew, but discount those payments by the renewal rate (i. e. the probability that they renew). Combining the discount due to renewal and the time-value of money discount, we arrive at the expression for the long term value of a contract.
\begin{equation}
    V = r_0\sum_{k=0}^{m_0 M N - 1} \beta^k x^n, \label{eq.ltv_pre_split}
\end{equation}
where
\begin{equation}
    n \equiv \left\lfloor\frac{k}{m_0 M} \right\rfloor  \nonumber
\end{equation}
is the the number of renewals that have happened by period $k$, $\lfloor . \rfloor$ represents the floor function, and
\begin{equation}
    r_0 \equiv \frac{V_0}{m_0} \nonumber
\end{equation}
is the payment amount.

\section{The LTV Equation}
We now take the steps required to turn the expression for recurring revenue, equation \ref{eq.ltv_pre_split}, into a closed form expression for LTV.  Our first step is to notice that $n=0$ for all $k < m_0 M$.  Similarly, $n=1$ for all $m_0 M <= k < m_0 M +1$, and so forth.  We can partition the sum of discounted revenue events into smaller sums, each having the same value for $n$ as follows:

\begin{align} \nonumber
    \frac{V}{r_0} &= \sum_{k=0}^{m_0 M N - 1} \beta^k x^n \\ \nonumber
    &= \sum_{k=0}^{m_0 M } \beta^k 
    + x \sum_{k=m_0 M}^{2 m_0 M } \beta^k 
    + x^2 \sum_{k=2 m_0 M}^{3 m_0 M } \beta^k 
    + \dots
    + x^{N - 1} \sum_{k=m_0 M \left(N-1\right)}^{m_0 M N - 1} \beta^k \\ \nonumber
    &= \sum_{k=0}^{m_0 M -1} \beta^k 
    + x \beta^{m_0M}\sum_{k=0}^{m_0 M-1 } \beta^k 
    + x^2 \beta^{2 m_0M}\sum_{k=0}^{m_0 M -1} \beta^k 
    + \dots
    + x^{N - 1} \beta^{m_0M (N-1)}\sum_{k=0}^{m_0 M -1} \beta^k \\ \nonumber
    &= \left(\sum_{k=0}^{m_0 M - 1} \beta^k\right)\left(\sum_{n=0}^{N-1} \left(x \beta ^{m_0 M}\right)^n\right).
\end{align}
But now recognizing the expressions for the geometric sum derived in the appendix, we can rewrite this as
\begin{equation}
    V = \frac{V_0}{m_0}
    \left(\frac{1 - \beta^{m_0 M}}{1 - \beta}\right)
    \left(\frac{1-\left(x \beta^{m_0 M}\right)^N}{1 - x \beta^{m_0 M}}\right),
\end{equation}  \label{eq.ltv_closed_from}
where
\begin{itemize}
    \item $V=$ the contract lifetime value (LTV)
    \item $V_0=$ the first year contract value (ACV)
    \item $m_0=$ the number of payments per year
    \item $M=$ the number of years in the contract
    \item $x=$ the contract renewal rate
    \item $\beta=\frac{1}{1 + \frac{i_0}{m_0}}=$ the discount fraction per billing period
    \item $i_0=$ the discount rate expressed as an annual rate
    \item $N=$ the total number of contract periods into the future to consider
\end{itemize}

\pagebreak

\section{The Case of Zero Discount Rate}
Let's simplify the LTV equation to describe the case where we ignore the discount rate, so $\beta=1$.
We have
\begin{align} \nonumber
    V &= \lim_{\beta \to 1}\left[\frac{V_0}{m_0}
    \left(\frac{1 - \beta^{m_0 M}}{1 - \beta}\right)
    \left(\frac{1-\left(x \beta^{m_0 M}\right)^N}{1 - x \beta^{m_0 M}}\right)\right] \\ \nonumber
    &= \frac{V_0}{m_0}
    \left(\frac{1 - x^N}{1 - x}\right)
    \lim_{\beta \to 1}
    \left[
    \left(\frac{1 - \beta^{m_0 M}}{1 - \beta}\right)
    \right] \nonumber \\
    &= \frac{V_0}{m_0}
    \left(\frac{1 - x^N}{1 - x}\right)
    \lim_{\beta \to 1}
    \left[
    \left(\frac{- m_0 M\beta^{m_0 M - 1}}{-1}\right)
    \right] \nonumber \\
    &= M V_0
    \left(\frac{1 - x^N}{1 - x}\right) \nonumber,
\end{align}  \label{eq.ltv_closed_from}
so

\begin{equation}
    V = M V_0 \left(\frac{1 - x^N}{1 - x}\right) \nonumber,
\end{equation} \label{eq.ltv_no_discount}
which makes sense.  The LTV is just total contract value $\left(M V_0\right)$ discounted by the renewal rate, $x$, for the duration of $N$ contracts. (Note that we used L'Hopital's Rule in taking the limit.)

\section{The Case of an Infinite Future}
Let's now consider the LTV of a situation where we don't stop considering our revenue stream after $N$ contract periods.  We just go off into the future indefinitely, so that $N$ becomes infinite.  In that case we have
\begin{align} \nonumber
    V &= \lim_{N \to \infty}\left[\frac{V_0}{m_0}
    \left(\frac{1 - \beta^{m_0 M}}{1 - \beta}\right)
    \left(\frac{1-\left(x \beta^{m_0 M}\right)^N}{1 - x \beta^{m_0 M}}\right)\right] \\ 
    V &= \frac{V_0}{m_0}
    \frac{1-\beta^{m_0 M}}{\left(1-\beta\right)\left(1-x\beta^{m_0 M}\right)}
\end{align}


\begin{appendices}

\pagebreak

\section{Geometric Sum}\label{section:first_appendix_section}
In computing quantities with recurring revenue, an expression that arises frequently is the geometric sum.
Let $r$ be some discount rate (say a renewal rate).  Then the sum of incrementally discounted rates is
\begin{equation}
    S_n = \sum_{k=0}^{n-1} ar^{k}.
\end{equation}

The numeric value for $S_n$ is obtained by considering the expressions
\begin{align}
    S_n &= a\left[1 + r + r^2 + \ldots  + r^{n-1} \right] \\
    r S_n &= a\left[0 +r + r^2 + \ldots + r^{n-1} + r^{n} \right].
\end{align}
Subtracting these two expression gives
\begin{equation}
    S_n - r S_n = a\left[ 1 - r^{n}\right].
\end{equation}
Rearranging gives the final expression
\begin{equation}
    S_n = \sum_{k=0}^{n-1} ar^{k} = a\left(\frac{1-r^n}{1-r}\right).
\end{equation}


\end{appendices}

% \printbibliography

%%% End document
\end{document}