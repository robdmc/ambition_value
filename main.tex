%%%%%%%%%%%%%%%%%%%%%%%%%%%%%%%%%%%%%%%%%%%%%%%%%%%%%%%%%%%%%%%%%%%%%% % LaTeX Example: Project Report
%
% Source: http://www.howtotex.com
%
% Feel free to distribute this example, but please keep the referral
% to howtotex.com
% Date: March 2011 
% 
%%%%%%%%%%%%%%%%%%%%%%%%%%%%%%%%%%%%%%%%%%%%%%%%%%%%%%%%%%%%%%%%%%%%%%
% How to use writeLaTeX: 
%
% You edit the source code here on the left, and the preview on the
% right shows you the result within a few seconds.
%
% Bookmark this page and share the URL with your co-authors. They can
% edit at the same time!
%
% You can upload figures, bibliographies, custom classes and
% styles using the files menu.
%
% If you're new to LaTeX, the wikibook is a great place to start:
% http://en.wikibooks.org/wiki/LaTeX
%
%%%%%%%%%%%%%%%%%%%%%%%%%%%%%%%%%%%%%%%%%%%%%%%%%%%%%%%%%%%%%%%%%%%%%%
% Edit the title below to update the display in My Documents
%\title{Project Report}
%
%%% Preamble
\documentclass[paper=a4, fontsize=11pt abstract]{scrartcl}

%% Uncomment this line to use custom margins
\usepackage[top=2.5cm, bottom=2.5cm, left=2.5cm, right=2.5cm]{geometry}

\usepackage[T1]{fontenc}
\usepackage{fourier}

\usepackage[english]{babel}															% English language/hyphenation
\usepackage[protrusion=true,expansion=true]{microtype}	
\usepackage{amsmath,amsfonts,amsthm} % Math packages
\usepackage[pdftex]{graphicx}	
\usepackage{url}
\usepackage[titletoc,title]{appendix}
\usepackage[boxruled ]{algorithm2e}
\usepackage{listings}
\usepackage{bm}
\usepackage{gensymb}
\usepackage{float}
\usepackage{pythonhighlight}
\usepackage{csquotes}

\usepackage{biblatex}
\addbibresource{sample.bib}


%%% Custom sectioning
\usepackage{sectsty}
%\allsectionsfont{\centering \normalfont\scshape}
\allsectionsfont{\normalfont\scshape}


%%% Custom headers/footers (fancyhdr package)
\usepackage{fancyhdr}
\pagestyle{fancyplain}
\fancyhead{}											% No page header
\fancyfoot[L]{}											% Empty 
\fancyfoot[C]{}											% Empty
\fancyfoot[R]{\thepage}									% Pagenumbering
\renewcommand{\headrulewidth}{0pt}			% Remove header underlines
\renewcommand{\footrulewidth}{0pt}				% Remove footer underlines
\setlength{\headheight}{13.6pt}

%%% Equation and float numbering
\numberwithin{equation}{section}		% Equationnumbering: section.eq#
\numberwithin{figure}{section}			% Figurenumbering: section.fig#
\numberwithin{table}{section}				% Tablenumbering: section.tab#


%%% Maketitle metadata
\newcommand{\horrule}[1]{\rule{\linewidth}{#1}} 	% Horizontal rule

\title{
		%\vspace{-1in} 	
		\usefont{OT1}{bch}{b}{n}
		\normalfont \normalsize \textsc{Models for Business} \\ [25pt]
		\horrule{0.5pt} \\[0.4cm]
		\huge Estimating the Value of Ambition\\
		\horrule{2pt} \\[0.5cm]
}
\author{
		\normalfont
		\normalsize
        Robert deCarvalho\\[-3pt]
        \normalsize
        \today
}
\date{}



%%% Begin document
\begin{document}
\maketitle


\begin{abstract}
A paper in which we quantify the value Ambition brings to your team.
\end{abstract}

%=============================================================================

\section{How Valuable is Ambition to My Team?}

At Ambition, we regularly get asked to estimate our ROI (Return on Investment).
Our customers want a way to put a dollar value on the benefits they receive from our product.
This question is almost impossible to answer in a one-size-fits-all kind of way.
Every team at every company is different.
The way they generate value is different.
The processes they follow are different.
Their workplace cultures are different.
You could even say that every team is its own special snowflake.

Over time we have carefully observed our customers to gather insights into our generated value.
This paper presents an overview of what we have learned.
We feel very confident in some of these insights.  
Others are still a work in progress.
We will be careful to separate what we know from what we think is probably true.


We hope this analysis will help you maximize Ambition's potential.
We also hope that it becomes a source of general knowledge for maximizing your team's efficiency.

\section{The Ambition Tool Set}
We curated Ambition's tools to target elements of your team's collaboration.  We put a lot of thought into ensuring that each targeted element will drive increased productivity.
Furthermore -- and perhaps more importantly -- we were very careful to construct a system whose value is more than the sum of it's parts.
Below is a list of the tools in the Ambition product.
As you read through them, see if you can identify the common themes.

\begin{itemize}
    \item \textbf{Scoring}
    Scoring is the heart of Ambition's mission.
    Managers configure a system that delivers a single number (score) to their employees.
    This number communicates exactly what the manager expects from them and how well they are meeting those expectation.
    By having two scores that individually target ``Activities'' and ``Objectives'', employees not only understand what the manager wants, but the exact path they should take to get there.
    \textbf{Scoring drives transparency. Transparency fosters honesty.  Honesty builds trust.}
    
    \item \textbf{Leader-boards, Accolades, TVs, Chat-Integration}
    Ambition believes that happy employees are productive employees.
    We believe that praise, recognition, and honest feedback increase employee satisfaction.
    We also recognize managers sink a lot of time and effort into doing these things well.
    So we automated it.
    With Ambition, managers configure multiple tools that automate praise, recognition and feedback.
    \textbf{Employees are thirsty for praise, and hungry for a sense of purpose.
    Ambition automates giving it to them.}
    
    \item \textbf{Playing Games}
    Ambition believes that ``play'' makes work more engaging.
    We believe people enjoy testing their abilities against their colleagues.
    We believe that spicing up the team with an enticing competition every now and then is exciting.
    We also know how much work it is to create and manage these competitions.
    So again, we automated it.
    Managers can configure a game to incentivize particularly desired behaviour.
    Configuring Ambition to do this is basically a set-it-and-forget-it proposition.
    \textbf{Games/competition create excitement and shape behaviour.}
    
    \item \textbf{Goals and Coaching}
    At Ambition we believe that quality one-on-one coaching is the most important job a manager has.
    A coach must understand an employee's blockers, teach them new processes, and refine their workflows.
    To do this well, both the coach and the employee need to have access to real time data.
    In talking to our customers, Ambition has discovered that preparing for effective one-on-ones is one the largest time-sucks a manager faces.
    So what did we do?
    We automated the busy-work.
    With Ambition's Coaching and Goals modules, managers conduct their one-on-ones with real-time, evidence-based context.
    Their preparation time is dramatically cut.
    Ambition keeps track of next-steps and follows up with accountability reminders.
    Manager time is freed up to do more impactful things.
    \textbf{Ambition brings evidence-based context into coaching sessions.}
   
   \item \textbf{Workflows, Reports, Notifications}.
   At Ambition, we understand that situational awareness is becoming increasingly important to managers.
   This is especially true in the time of increased remote work.
   We believe the best managers know how each team member is performing ``right-now.''
   We understand that it takes work to dig through your CRM for the information you need.
   We understand that you'd like to get notified when a specific thing happens to an individual employee.
   We built a system that notifies you when events that you care about happen.
   Our system allows you to configure and save reports that target exactly the information you need to know ``right now.''
   \textbf{Ambition makes real-time situational awareness easy.} 
    
\end{itemize}

\section{The Structure of Ambition's Value}
Having read through Ambition's feature set, our three main themes should be obvious:
\begin{itemize}
    \item Save people time.
    \item Make people better.
    \item Make people happier.
\end{itemize}

We will quantify the value each of these generates for your team.

\subsection{Save People Time}
Ambition's time-saving automation is targeted at managers.
We aim to reduce the busy-work required to be a good manager.
After adopting ambition, a manager's workload will fall to a fraction of it's previous value.
This decreased workload per employee increases the number of people a manager can supervise.
Based on internal polling, we know that Ambition saves managers between 7 and 9 hours of work per week.
This is a 20\% increase in capacity.
There are several ways to estimate the dollar value of manager-time saved.
We opt for the simplest.
We value that saved time at the fully loaded hourly rate for that manager.
\textbf{Ambition automation reduces your management overhead cost by approximately 20\%.}

\subsection{Make People Better}
We promised we would clearly delineate between ``things we know'' and ``things we think are probably true.''
This section falls squarely into the second bucket.
Making claims about cause and effect is notoriously hard.
Do we think Ambition is improving people's performance?  Yes!
Can we be sure that it's Ambition instead of a myriad other things we don't even know are happening at your company?  No!
To do this, we'd need  a customer (do you want to volunteer?) to design a randomized trial.
Until then, we are relegated to anecdotes and plausibility arguments.
We have thought of many ways that Ambition plausibly boosts performance, and highlight the three main contenders here.
\begin{itemize}
    \item When managers, fellow employees and even the occasional VP shower praise on an individual, that employee feels amazing.
    Seeking a repeat of this very-public approval makes people do more and be better.
    
    \item The score building process is often transformative for managers.
    It forces them to quantitatively describe what it takes for an employee to be better.
    Although this seems obvious, you'd be surprised at how often we encounter managers who have real difficulty with this.
    Ambition forces them to carefully and quantitatively evaluate their existing processes.
    
    \item Ambition's tooling often enables processes that were simply not feasible before.
    Managers end up with more space to be imaginative.
    They end up creating value in ways they would never have thought of before Ambition.
\end{itemize}

So how should you put a dollar value on these improvements?
We don't know.
We have, however, developed an intuition by working with our customers over the past several years.
Some of them have seen tremendous performance boosts.
Others, none at all.
We believe this variance is driven by how well the teams actually use Ambition's tools.
So, without knowing anything about your team, what's our guess on how much Ambition will boost your performance?
\textbf{Our guess is that Ambition will boost your veteran employee's production by at least half a percent, and much more if you use it well.}

\subsection{Make People Happier}
Companies that buy Ambition are typically facing employee-retention challenges.
A core value of Ambition is making people happier at work so they don't leave.
This generates measurable value.

Let's start with the obvious.  Every time someone leaves, you need to hire a replacement.
This presents direct costs for recruiting, hiring, and training.
There are, however, more nuanced costs associated with churn.
Every time you lose a veteran producer, the vacancy is filled with an inexperienced replacement.
Until that rookie is trained up into a veteran, your company is suffering an opportunity cost.
This cost is equal to the performance differential between rookies and veterans.
The true cost of employee churn then then the sum of two things:
How much it costs to replace an employee, and the opportunity cost of their inexperience.

These costs are calculable.
Based on observing our customers, we are confident that Ambition will increase your employee-retention time and decrease your training time.
\textbf{We estimate that improvements Ambition makes to training and retention result in about a 0.5\% increase of your team's output.}

\subsection{Putting it all Together}
We have described how Ambition generates value.
In each section, we put numbers to paper and included our good-faith estimate of how big that value is.
\textbf{We don't, however, expect you to take our word for it.}
We expect that the unique way your team operates is different from the broad assumptions we made above.
To this end, we have derived straightforward formulas you can use in your own spreadsheets.
These derivations comprise the remainder of this paper.
Additionally, \textbf{we can provide you with a spreadsheet pre-populated with our math that will let you compute your own version of Ambition's value.}

\section{Quantitative Formulas for Ambition Value}
This section will develop the math for computing Ambition's value in a spreadsheet.
It will be heavy on the algebra.
We will present formulas first, and derive them later.
This should enable you to get what you need while minimizing your exposure to the horror that is math.

\subsection{Definitions}
In order to think clearly about the valuation problem, we translate the various quantities into algebraic variables which are defined here.
\begin{itemize}
    \item $Q_0$\textbf{: The total value generated by the team (without Ambition)}
    This is an estimate (in dollar terms) of the total annual value generated by this team prior to adopting Ambition.
    
    \item $Q$\textbf{: The total value generated by the team (with Ambition)}
    This is an estimate (in dollar terms) of the total annual value generated by this team after adopting Ambition.
    
    \item $N_p$\textbf{: The total number of producers on the team.}
    This is the number of people on the team that actually do the work.  It does not include managers or support staff.
    
    \item $N_m$\textbf{: The total number of managers on the team.}  This is the number of managers on the team.
    
    \item $N_r$\textbf{: The number or ``rookies'' on the team.}  This is the number of producers that are not yet fully trained.  Until they are fully trained, their performance is degraded in comparison to their veteran colleagues.
    
    \item $N_v$\textbf{: The number or ``veterans'' on the team.}  This is the number of fully-trained producers on the team. They are assumed to achieve full-performance.
    
    \item $q_{v}$\textbf{: The pre-ambition annual value generated by a ``veteran.''}  This is the amount of value we expect fully-trained veteran to generate if they did not use Ambition.
    
    \item $q_{n}$\textbf{: The pre-ambition net veteran value}  This is the value that a veteran generates for the company net of what it costs to keep them on payroll.
    
    \item $\alpha$\textbf{: The ``rookie'' discount rate.}  Rookies only generate this fraction of what a fully trained veteran can produce.
    
    \item $T_c$\textbf{: Time to churn} This is the average length of time an employee stays with the company
    
    \item $T_t$\textbf{: Time to train} This is time it takes to train a new hire into a veteran.
    
    \item $\gamma$\textbf{: The churn-to-train ratio prior to adopting Ambition} The ratio of churn-time (a.k.a. employee-retention-time) to train-time is a fundamental value-driver.  This is that ratio prior to introducing Ambition.
    
    \item $\gamma_a$\textbf{: The churn-to-train ratio after adopting Ambition} The churn-to-train ratio after incorporating Ambition's improvements.
    
    \item $\chi_L$\textbf{: After using ambitions, veterans improve by this fraction.} This the ratio by which Ambition improves veteran performance.
    
    \item $\Delta_L$\textbf{: The value change caused by Ambition improving veteran performance} Your team will realize this much more value from the performance bump Ambition creates in veteran employees.
    
    \item $\chi_a$\textbf{: Automation time factor} Engaging Ambition's automation capability reduces a managers required work hours by this fraction.
    
    \item $\chi_c$\textbf{: Churn time factor} Using Ambition increases your churn time by this factor.
    
    \item $\chi_t$\textbf{: Train time factor} Using Ambition decreased your training time by this factor.
    
    \item $c_m$\textbf{: The total loaded cost of a single manager} This is how much it costs to keep a single manager on staff.
    
    \item $c_p$\textbf{: The total loaded cost of a single producer} This is how much it costs to keep a single producer on staff.
    
    \item $c_h$\textbf{: Hiring/recruiting cost for one producer} This is how much it costs to recruit and hire a single producer.
    
    \item $t_h$\textbf{: Time to hire} This is how long a churned position stays vacant.
    
    \item $\Delta_a$\textbf{: The value change caused by Ambition automation.} This is the dollar impact of Ambition's automation.
\end{itemize}


\subsection{The Value of Saving People Time}
Computing the value of saving managers time is easy.
We make managers more efficient by automating busy work.
This enables them to manage a larger team.
The savings to your management overhead cost is simply the saved time valued at the fully-loaded manager hourly rate.
\begin{equation}
    \Delta_a = N_m c_m \left(1 - \chi_a\right)
\end{equation}


\subsection{The Value of Making People Better}
We can only make an informed guess on how much Ambition improves veteran performance.
Once we have made that guess though, the computation is easy.
\begin{equation}
    \Delta_L = Q_0\left(\chi_L - 1\right)
\end{equation}


\subsection{The Value of Making People Happier}
This may not end up being the most important value driver, but it is the most involved to calculate.
The subsections below go into the derivation.

The first thing we care about is estimating \textbf{the value generated by a single veteran producer prior to adopting ambition.}

\begin{equation}
    q_{v} = \frac{Q_0}{N_p}\left(\frac{1 + \gamma}{\alpha + \gamma}\right).
\end{equation}

We can then use the veteran production value to obtain two equivalent expressions for \textbf{the total team production value after adopting Ambition.}
\begin{align}
    Q &= \chi_L q_{v} N_p\frac{\left(\frac{\alpha}{\chi_L}\right) + \gamma_a}{1 + \gamma_a} \label{Q0} \\
    Q &= \chi_L Q_0  \left(\frac{1+\gamma}{1 + \gamma_a}\right) \left(\frac{\frac{\alpha}{\chi_L} + \gamma_a  }{\alpha + \gamma}\right).
\end{align}

Finally, we compute \textbf{the savings realized by reducing hiring/training cost.}
\begin{align}
    \Delta_h = \frac{c_h N_p}{T_c}\left(1 - \frac{1}{\chi_c}\right)
\end{align}

\subsubsection{Model Setup}
We will make the following assumptions about your team.  It consists of $N_p$ producers being managed by $N_m$ managers.  The average producer stays with your company for a duration of $T_c$ (we call this time-to-churn).  It takes a time $T_t$ (time-to-train) to fully ramp a new hire transforming them from a rookie to a veteran. The total annual value generated by your team prior to adopting Ambition is $Q_0$, and the fully loaded cost of a manger is $c_m$.  When people churn, it costs $c_h$ to recruit and hire their replacement.

\subsubsection{Computing time-to-churn}
You may not immediately know how long the average employee stays with the company, so this section gives you a quick way to estimate it given numbers you probably know off the top of your head.

Let's say that on a team of $N_p$ producers, you churn $N_c$ of them in some period, $\tau$.  So, for example, you could churn 6 producers a year on your team of 20 people.  In this case, $N_p=20$, $N_c=6$ and $\tau=1\text{ year}$. If we assume that the churn events are completely random, this becomes a classic problem in survival analysis -- that of a constant hazard function. All this means is that we are justified in estimating the time-to-churn using the following formula
\begin{equation}
    T_c = \tau \frac{N_p}{N_c}.
\end{equation}
So, for our example,
\begin{align}
    T_c &= \left(\text{1 year}\right)\frac{20}{6} \\
    T_c &= 3.33\text{ years}.
\end{align}

\subsubsection{Computing Rookie/Veteran Proportions}
Intuitively we know that our team will be composed mostly of veterans when our churn rate is really low.  Similarly, if our churn rate is really high, we know that our team is going to be mostly composed of rookies.  Let's put some effort into making this more quantitative.

Let's think about how people transition through the system.  The rate at which producers churn from the team is
\begin{equation}
    R_c = \frac{N_p}{T_c}
\end{equation}
If our team is to remain at the same size, we need to hire at the same rate as people churn so that means our hiring rate must be equal to our churn rate, or
\begin{equation}
    R_h = R_c = \frac{N_p}{T_c}.
\end{equation}

Now, we said that it takes some time, $T_t$ to fully ramp producers from rookies to veterans.  That means that the rate at which we lose rookies because they've been transformed to veterans is
\begin{equation}
    R_t = \frac{N_r}{T_t}.
\end{equation}
Furthermore, we can separately compute the rate at which rookies and veterans churn
\begin{align}
    R_r &= \frac{N_r}{T_c} \\
    R_v &= \frac{N_v}{T_c}.
\end{align}

Now lets put all this together.  If our team's composition is not changing, this means that the rate at which we hire people must be balanced by the the rate at which we churn rookies and the rate at which we transform them into veterans, or
\begin{align}
    R_h &= R_r + R_t \\
    \frac{N_p}{T_c} &= \frac{N_r}{T_c} + \frac{N_r}{T_t} \\
    N_p &= N_r\left(1 + \frac{T_c}{T_t}\right) \\
    N_r &= \frac{N_p}{\left(1 + \frac{T_c}{T_t}\right)} = \frac{N_p}{\left(1 + \gamma\right)} \label{Nr}
\end{align}
Similarly, the rate at which we churn veterans must be equal to the rate at which create them if our team has a stable composition.  So
\begin{align}
    R_t &= R_v \\
    \frac{N_r}{T_t} &= \frac{N_v}{T_c} \\
    N_v &= \frac{T_c}{T_t}N_r = \gamma N_r \\
    N_v &= \frac{\gamma N_p}{\left(1 + \gamma\right)} \label{Nv}
\end{align}

Equations \ref{Nr} and \ref{Nv} are the stable proportions of veterans and rookies given the ratio of churn to train time.

\subsubsection{Computing Value as a function of Churn-to-Train Time Ratio}
Recall that we assumed that rookies generated value only at a rate of $\alpha$ compared to our veterans.  This means, that if your team's total production prior to adopting Ambition was $Q_0$, the breakdown of how that came from rookies and veterans must have been
\begin{equation}
    Q_0 = \alpha q_{v} N_r + q_{v} N_v = q_{v}\left(\alpha N_r + N_v\right).
\end{equation}
Substituting our previously obtained expressions for the rookie/veteran populates, we obtain
\begin{align}
    Q_0 &= q_{v}\left(\frac{\alpha N_p}{1 + \gamma} + \frac{\gamma N_p}{1 + \gamma}\right)  \\
    Q_0 &= q_{v} N_p\frac{\alpha + \gamma}{1 + \gamma} \label{Q0}.
\end{align}
Let's take a look at this equation because it's interesting.  think about what happens when your churn time is much much longer than your train time.  In this case $\gamma >> 1$ and your total production will approach $q_{v}N_p$. This would be like treating every producer like a veteran.  This makes sense because most of your producers WILL be veterans under these circumstances.  Now consider the case where your churn time is much smaller than your train time.  This would reflect the case where you lose people before you can fully train them.  In this case $\gamma << 1$ and your total production will approach $\alpha q_{v} N_p$.  This is the production you would expect if you had a team full of rookies all having reduced production.

We can obtain one more useful relation if we invert Equation (\ref{Q0}).  Solving for $q_{v}$ gives us an expression for the veteran-generated revenue prior to Ambition being adopted,
\begin{equation}
    q_{v} = \frac{Q_0}{N_p}\left(\frac{1 + \gamma}{\alpha + \gamma}\right).
\end{equation}

Now to pull it all together by thinking about how Ambition impacts each variable in the equation.  Ambition will not impact the skill of incoming rookies, but it will elevate the performance of veterans by a factor of $\chi_L$.  This means that the rookies will have a steeper hill to climb, and post ambition their performance reduction will actually be $\alpha / \chi_L$.  Furthermore the churn-to-train time ratio will go from $\gamma$ to $\gamma_a$.  Making these three substitutions in Equation (\ref{Q0}), we obtain an expression for the team production after adopting Ambition.
\begin{align}
    Q &= \chi_L q_{v} N_p \frac{\left(\frac{\alpha}{\chi_L}\right) + \gamma_a}{1 + \gamma_a} \\
    Q &= \chi_L \left(     \frac{Q_0}{N_p}\left(\frac{1 + \gamma}{\alpha + \gamma}\right)             \right) N_p \frac{\left(\frac{\alpha}{\chi_L}\right) + \gamma_a}{1 + \gamma_a} \\
    Q &= \chi_L Q_0  \left(\frac{1 + \gamma}{\alpha + \gamma}  \right)  \frac{\left(\frac{\alpha}{\chi_L}\right) + \gamma_a}{1 + \gamma_a} \\
    Q &= \chi_L Q_0  \left(\frac{1+\gamma}{1 + \gamma_a}\right) \left(\frac{\frac{\alpha}{\chi_L} + \gamma_a  }{\alpha + \gamma}\right)
\end{align}

NEED TO UPDATE THIS, BUT IN COMPUTING SPREADSHEET I CAME UP WITH THIS VALUE FOR THE ``RAMPING COST.''
BASICALLY THE DIFFERENTIAL YOU ARE PAYING BY HAVING ROOKIES INSTEAD OF VETS.

** I need to confirm this. **

\begin{equation}
    C_r = q_v N_p \frac{1 - \alpha}{1 + \gamma}
\end{equation}

\subsubsection{Recruiting/Hiring Savings}
We assume you want to maintain a constant team size.
To do this you will need to hire at the same rate as you churn employees.
\begin{equation}
    R_h = R_c = \frac{N_p}{T_c} 
\end{equation}
Ambition increases your employee retention rate.  This lowers your hiring rate.
Each new hire costs you $c_h$ to recruit and hire.
This give the following saving in hiring cost.
\begin{align}
    \Delta_h = \frac{c_h N_p}{T_c}\left(1 - \frac{1}{\chi_c}\right).
\end{align}


In addition to the hiring cost, we must consider the opportunity cost of the time it takes to hire a replacement.  Let's call this the change in vacancy cost.
\begin{equation}
    \Delta_v = \left(q_v - c_p\right)\left(1-\frac{1}{\chi_c}\right)\frac{t_h}{1 \textrm{year}}
\end{equation}

% \printbibliography

%%% End document
\end{document}